% Created 2018-11-01 to. 20:49
\documentclass{article}
\usepackage[utf8]{inputenc}
\usepackage[T1]{fontenc}
\usepackage{fixltx2e}
\usepackage{graphicx}
\usepackage{longtable}
\usepackage{float}
\usepackage{wrapfig}
\usepackage{rotating}
\usepackage[normalem]{ulem}
\usepackage{amsmath}
\usepackage{textcomp}
\usepackage{marvosym}
\usepackage{wasysym}
\usepackage{amssymb}
\usepackage{hyperref}
\tolerance=1000
\date{}
\title{}
\hypersetup{
  pdfkeywords={},
  pdfsubject={},
  pdfcreator={Emacs 24.5.1 (Org mode 8.2.10)}}
%\renewcommand\refname{Referanser}
\begin{document}


\section*{Problemstilling}
\label{sec-1}
Min problem stilling går ut ifra en dobbelttime gjennomført på kongshavn vid regående skole under praksis.
Timen er ment for en helklasse i R2. Jeg har har valgt å fokusere på hvordan redskaper som 
digitale verktøy  og rike oppgavene kan påvirke undervisningen og hjelpe elevene få en aha-opplevelse.
Tidligere har elevene gått igjennom vektorer i rommet for første gang, der ble det benyttet geogebra
for å definere tre-dimensjonale objekter.\\
Igjennom oppgavene og teorien vi gikk igjennom da, fikk eleven kjenne på hvordan abstrakte konsepter de 
kjente fra før i dimensjon to slik som ortogonalitet og vinkler ble vanskeligere i visualisere.\\
Dette blir repetert igjen i begynnelsen av denne timen. Siden skalarprodukt og vektor produkt er abstrakte konstruksjoner som bygger på vektorer i rommet,
 vil visualisering ved hjelp av dataverktøy være en viktig resurs Oppgavene som gis til elevene vil også vektlegge den geometrisk tolkningen av skalarprodukt og kryssprodukt.\\

%Vi ønsker å kontekstualisere det vi observerte i timene med teoriene vi finner \cite{digital} om digitale læreverktøy i skolen.
%Inneholdet i timene er den mer abstrakte algebraen man lærer i vidregående skole og vil derfor gi oss en utfording i å reflektere rundt det slik som regnefortellinger man finner i  \cite{alg}.
\section*{Hvorfor denne problemstillingen?}
Jeg har valgt å skrive oppgave om denne problemstillingen siden dobbeltime matte kan by på flere utfordringer feks. som hvordan man holder fokuset og interessen til elevene.
I tilegg til dette vil jeg undersøke hvordan man kan formidle abstrakte algebraiske konsepter og vekke interesse for feltet samt se hva som er en interessant oppgave knyttet til tema.
\newpage
\section*{Timeplaner}
\label{sec-2}
\subsection*{Første time}
\label{sec-2-1}
\begin{center}
\begin{tabular}{l|l|l|l}
Tid & Hva skjer? & Hvordan skal & Hvorfor skal\\
 &  & dette skje? & det skje?\\
\hline
0-3 & Oppstart av timen. & Få ro i klassen. & For å få oppmerksomheten\\
min &  & Elevene setter seg & og ro hos elevene.\\
 &  & faste sitteplasser. & \\
 &  & Ta fravær ved at & \\
 &  & elevene krysser seg & \\
 &  & av. & \\
\hline
3-15 & Gjennomgang av temaet. & Elevene lytter og de & Dette skal skjer for\\
min & Læreren repiterer & som vil tar notater. & å formidle temaet.\\
 & tidligere materiale, & Først vil noen & Læreren vil stille\\
 & samt presentere & minutter bli satt & spørsmål på kritiske\\
 & elevene for & av tid til å snakke & punkter for å gi dem\\
 & skalarproduktet. & om vektorer og hva & en bedre forståelse\\
 &  & det vil si at vi har & samt skape dialog\\
 &  & ett produkt av & i klassen. Deretter\\
 &  & vektorer. Deretter & forklares det hvordan\\
 &  & vil skalarproduktet & skalarproduktet vi\\
 &  & bli innført og & innfører nå, likner på\\
 &  & egenskapene til dette & skalarproduktet i\\
 &  & produktet bli forklart. & planet.\\
 &  & Så vil en & \\
 &  & eksempeloppgave & \\
 &  & bli gjennomgått. & \\
\hline
15-35 & Elevene arbeider med & Eleven jobber med & Dette gjøres for å\\
min & oppgaver de har ifra & oppgavene. Hvis noen & styrke elevens forståelse\\
 & ett ark Læreren har & ikke forstår oppgaven & av tema et som har blitt\\
 & lagd. Læreren går & eller trenger hjelp, & gjennomgang av læreren.\\
 & rundt og hjelper & rekker de opp hånda. & Dette gir også læreren\\
 & elevene med oppgavene. &  & mulighet til å gjengi noe\\
 &  &  & eleven ikke har forstått,\\
 &  &  & eller som  var uklart.\\
\hline
35-45 & Læreren vil gjennomgå & Læreren gjennomgår & Dette gjøres for\\
min & løsning på noen av de & oppgaver som elevene & konsolidering og gir i\\
 & oppgavene elevene & har slitt med eller de & tillegg en mulighet for\\
 & slet med. Samt gå & oppgavene som var mest & å oppklare noe elevene\\
 & igjennom hva timen & instruktive. Elevene & ikke har forstått.\\
 & handlet om. & vil fortsette å arbeide & \\
 &  & med tidligere oppgaver, & \\
 &  & dersom de ikke ønsker å & \\
 &  & følge gjennomgangen. & \\
\end{tabular}
\end{center}
\newpage
\subsection*{Andre time}
\label{sec-2-2}
\begin{center}
\begin{tabular}{l|l|l|l}
Tid & Hva skjer? & Hvordan skal & Hvorfor skal\\
 &  & dette skje? & det skje?\\
\hline
0-2 & Oppstart av timen. & Få ro i klassen etter & For å få oppmerksomheten\\
min &  & friminutt. Læreren vil & og roen hos elevene.\\
 &  & få overblikk over hvem & \\
 &  & som er tilstedet. & \\
\hline
2-15 & Introduksjon av & Etter elevene har & Dette gjøre for å\\
min & kryssprodukt. & fått roen, kan & formidle stoffet til\\
 &  & læreren gå igjennom & elevene. Spesielt viktig\\
 &  & prinsippene til & blir da den så kalte\\
 &  & kryssproduktet. & høyre håndsregelen.\\
 &  & Slik som hvordan & Denne blir sentral i\\
 &  & kryss produktet & den geometriske\\
 &  & gir en vektor fra & tolkningen av\\
 &  & to vektorer, & kryssproduktet.\\
 &  & i motsetning til & \\
 &  & skalarproduktet som & \\
 &  & gir ett tall. & \\
 &  & Deretter vil det bli & \\
 &  & gjennomgått hvilke & \\
 &  & egenskaper dette & \\
 &  & produktet har, og den & \\
 &  & geometriske tolkningen. & \\
\hline
15-35 & Arbeid med oppgaver. & Eleven jobber med & Oppgavene vil fokusere\\
min &  & oppgavene. Hvis noen & på regneregler og\\
 &  & ikke forstår oppgaven & determinant metode. I\\
 &  & eller trenger hjelp, & tilegg til formler for\\
 &  & rekker de opp hånda. & areal av trekanter.\\
\hline
35-45 & Oppgaver løst på & Læreren gjennomgår & Dette gjøres for\\
min & tavlen. & oppgaver som elevene & konsolidering og gir i\\
 &  & har slitt med eller & tillegg en mulighet for\\
 &  & oppgavene som var me & å oppklare noe elevene\\
 &  & instruktive. Elevene & ikke har forstått.\\
 &  & vil fortsette å arbe & \\
 &  & med tidligere oppgav & \\
 &  & dersom de ikke ønske & \\
 &  & følge gjennomgangen. & \\
 &  &  & \\
\end{tabular}
\end{center}
% Emacs 24.5.1 (Org mode 8.2.10)
%\newpage
%\begin{thebibliography}{2}
%\bibitem[Heid 2013]{digital}
%Heid, M. K., Thomas, M. O. J.,  Zbiek, R. M. 
%\newblock How Might Computer Algebra Systems Change the Role of Algebra in the School Curriculum?
%\newblock { \em Third International Handbook of Mathematics Education} 2013
%
%\bibitem[Har 2009]{alg}
%Har, Y. B. (2009). 
%\newblock Teaching of algebra. I L.P. Yee \& L.N..  
%\newblock {Teaching secondary school mathematics. A resourse book},  25 – 50. Singaporte: McGraw Hill.
%
%\bibitem[Grevholm 2003]{grev}
%Barbro Grevholm.
%\newblock Matematikk for skolen.
%\newblock {\em Fagbokforlaget}
%
%\bibitem[Perspectives on learning 2004]{learning}
%International Perspectives on learning and Teaching Mathematics
%\newblock {\em Gøteborg University} 
%\end{thebibliography} 
%
\end{document}
