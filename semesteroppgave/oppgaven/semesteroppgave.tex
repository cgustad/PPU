% Created 2018-11-21 on. 17:00
\documentclass[11pt]{article}
\usepackage[utf8]{inputenc}
\usepackage[T1]{fontenc}
\usepackage{fixltx2e}
\usepackage{graphicx}
\usepackage{longtable}
\usepackage{float}
\usepackage{wrapfig}
\usepackage{rotating}
\usepackage[normalem]{ulem}
\usepackage{amsmath}
\usepackage{textcomp}
\usepackage{marvosym}
\usepackage{wasysym}
\usepackage{amssymb}
\usepackage{hyperref}
\usepackage{cite}
%\usepackage[numbers]{natbib}
\usepackage[round]{natbib}


\usepackage{natbib}
\bibliographystyle{abbrvnat}
\setcitestyle{authoryear,open={[},close={]}}
\tolerance=1000
%\bibliography{semsteroppgave}
\usepackage[margin=1.0in]{geometry} \setlength\parindent{0pt}
\author{Christian O'Cadiz Gustad}
\date{\today}
\title{Semesteroppgave}
\hypersetup{
  pdfkeywords={},
  pdfsubject={},
  pdfcreator={Emacs 25.2.2 (Org mode 8.2.10)}}

\begin{document}

\maketitle

\section*{Problemstillling semesteroppgaven}
\label{sec-1}
Min problemstilling går ut ifra en dobbelttime gjennomført på kongshavn vidregående skole under praksis.
Timen er ment for en helklasse i R2. Jeg har har valgt å fokusere på hvilke redskaper som 
digitale verktøy og illustrende oppgaver kan påvirke undervisningen og hjelpe elevene få en aha-opplevelse.   


\newpage
%\section*{Timeplaner}
%\label{sec-2}
%\begin{center}
%\begin{tabular}{l|l|l|r}
%Dato & Klasse & Varighet & Tidspunkt\\
%1.Nov & R2 & To timer & 13:00-14:30\\
%\hline
%Fag: & Kompetansemål:\\
%Matematikk & \\
% & \\
% & \\
%\hline
% &  &  & \\
% &  &  & \\
% &  &  & \\
% &  &  & \\
%\end{tabular}
%\end{center}
\subsection*{Første time}
\label{sec-2-1}
\begin{center}
\begin{tabular}{l|l|l|l}
Tid & Hva skjer? & Hvordan skal & Hvorfor skal\\
 &  & dette skje? & det skje?\\
\hline
0-3 & Oppstart av timen. & Få ro i klassen. & For å få oppmerksomheten\\
min &  & Elevene setter seg & og roen til elevene.\\
 &  & faste sitteplasser. & \\
 &  & Ta fravær ved at & \\
 &  & elevene krysser seg & \\
 &  & av. & \\
\hline
3-15 & Gjennomgang av temaet. & Elevene lytter og de & Dette skal skjer for\\
min & Læreren repiterer & som vil tar notater. & å formidle temaet.\\
 & tidligere matriale, & Først vil noen & Læreren vil stille\\
 & samt presentere & minutter bli satt & spørsmål på kritiske\\
 & elevene for & av tid til å snakke & punkter for å gi dem\\
 & skalarproduktet. & om vektorer og hva & en bedre forståelse\\
 &  & det vil si at vi har & samnt skape dialog\\
 &  & ett produkt av & i klassen. Deretter\\
 &  & vektorer. Deretter & forklares det hvordan\\
 &  & vil skalarproduktet & skalarproduktet vi\\
 &  & bli innført og & innfører nå, likner på\\
 &  & egenskapene til dette & skalarproduktet i\\
 &  & produktet bli forklart. & planet.\\
 &  & Så vil en & \\
 &  & eksempeloppgave & \\
 &  & bli gjennomgått. & \\
\hline
15-35 & Elevene arbeider med & Eleven jobber med & Dette gjøres for å\\
min & oppgaver de har ifra & oppgavene. Hvis noen & styrke elevens forståelse\\
 & ett ark Læreren har & ikke forstår oppgaven & av temaet som har blitt\\
 & lagd. Læreren går & eller trenger hjelp, & gjennomgang av læreren.\\
 & rundt og hjelper & rekker de opp hånda. & Dette gir også læreren\\
 & elevene med oppgavene. &  & mulighet til å gjengi noe\\
 &  &  & eleven ikke har forstått,\\
 &  &  & eller som  var uklart.\\
\hline
35-45 & Læreren vil gjennomgå & Læreren gjennomgår & Dette gjøres for\\
min & løsning på noen av de & oppgaver som elevene & konsolidering og gir i\\
 & oppgavene elevene & har slitt med eller de & tillegg en mulighet for\\
 & slet med. Samt gå & oppgavene som var mest & å oppklare noe elevene\\
 & igjennom hva timen & instruktive. Elevene & ikke har forstått.\\
 & handlet om. & vil fortsette å arbeide & \\
 &  & med tidligere oppgaver, & \\
 &  & dersom de ikke ønsker å & \\
 &  & følge gjennomgangen. & \\
\end{tabular}
\end{center}
\newpage
\subsection*{Andre time}
\label{sec-2-2}
\begin{center}
\begin{tabular}{l|l|l|l}
Tid & Hva skjer? & Hvordan skal & Hvorfor skal\\
 &  & dette skje? & det skje?\\
\hline
0-2 & Oppstart av timen. & Få ro i klassen etter & For å få oppmerksomheten\\
min &  & friminutt. Læreren vil & og roen til elevene.\\
 &  & få overblikk over hvem & \\
 &  & som er tilstedet. & \\
\hline
2-15 & Inroduksjon av & Etter elevene har & Dette gjøre for å\\
min & kryssprodukt. & fått roen, kan & formidle stoffet til\\
 &  & læreren gå igjennom & elevene. Spesielt viktig\\
 &  & prinsippene til & blir da den så kalte\\
 &  & kryssproduktet. & høyrehåndsregelen.\\
 &  & Slik som hvordan & Denne blir sentral i\\
 &  & kryss produktet & den geometriske\\
 &  & gir en vektor ffra & tolkningen av\\
 &  & to vektorer, & kryssproduktet.\\
 &  & i motsetning til & \\
 &  & skalarproduktet som & \\
 &  & gir ett tall. & \\
 &  & Deretter vil det bli & \\
 &  & gjennomgått hvilke & \\
 &  & egenskaper dette & \\
 &  & produktet har, og den & \\
 &  & geometriske tolkningen. & \\
\hline
15-35 & Arbeid med oppgaver. & Eleven jobber med & Oppgavene vil fokusere\\
min &  & oppgavene. Hvis noen & på regneregler og\\
 &  & ikke forstår oppgaven & determinant metode. I\\
 &  & eller trenger hjelp, & tilegg til formler for\\
 &  & rekker de opp hånda. & areal av trekanter.\\
\hline
35-45 & Oppgaver løst på & Læreren gjennomgår & Dette gjøres for\\
min & tavlen. & oppgaver som elevene & konsolidering og gir i\\
 &  & har slitt med eller & tillegg en mulighet for\\
 &  & oppgavene som var me & å oppklare noe elevene\\
 &  & instruktive. Elevene & ikke har forstått.\\
 &  & vil fortsette å arbe & \\
 &  & med tidligere oppgav & \\
 &  & dersom de ikke ønske & \\
 &  & følge gjennomgangen. & \\
 &  &  & \\
\end{tabular}
\end{center}
\newpage
\section{Beskrivelse av klassen og presentasjonen av undervisningsopplegget.}
\label{sec-1}

Praksisskolen er en videregående skole i Oslo, som er fokusert på studiespesialserende men har også tilbud innen idrett og entreprenørskap. Klassen består av 14 elever som har valgt seg inn til faget R2 matematikk. Siden R2 er ett programfag vil en høyere andel av elevene være motiverte til å lære faget. Dette har jeg også fått intrykk av under tidligere observasjonstimer hos klassen. 

Kjønnsfordelingen i klassen er jevnt, og man får inntrykk av at jentene ligger jevnt over på ett høyere faglig nivå enn guttene. Spesielt er det en gjeng på 2-3 gutter som ikke gjør arbeidet de skal i de tidligere timene vi har observert. Ellers virker klassemiljøet godt der det er liten annen problematferd. Det virker som om elevene har god relasjon til læreren og til hverandre.\\

Derfor vil disse elvente være spesielt motiverte og vil være i stor grad derevet av indre motivasjon \cite{KletteH}. Dette gir læreren frihet og rom i undervisningen og kan gå igjennom ulike perspektiver på temaet.

I denne semestroppgaven vil jeg først beskrive oplegget samnt analysere og drøfte opplegget og resultatet ut fra pedagogisk og fagdidaktikks teori.

\section{Presentasjon av undervisning opplegget}
\label{sec-2}
Den faste underviseren deres er en svært god lærer som har gitt ut flere bøker innen matematikk for videregående. Terskelen virker lav for at elevene skulle  komme med inspill og spørsmål til undervisningen. 
Strukturen på undervisningen hennes virker å gå som følger:
\begin{itemize}
\item \textbf{(7-15 min)} Gå gjennom teori og eksempler på tavla.
\item \textbf{(20-28 min)} Elevene får arbeide med oppgaver alene.
\item \textbf{(10 min)} Læreren går igjennom vansklige og illustrerende oppgaver på tavla.
\end{itemize}
Vi bestemte oss derfor å holde oss til denne tidsplanen, men denne gangen med i en dobbeltime. 
Målet for den første undervisningstimen var å gjøre elevene kjent med skalarproduktet i dimensjon tre. Deretter introdusere vektorproduktet i dimensjon tre, som vil være helt nytt i motsetning til skalarproduketet som elevene kjenner fra tidligere i dimensjon to.\\

Før denne timen har vi observert flere timer som læreren har holdt, samt holdt en dobbeltime der vi gikk igjennom vektorer og konstruksjoner i dimensjon tre.\\
Etter  timene under veiledning av denne læreren var det noen punkter hun anbefalte at jeg forbedret meg på nemlig:
\begin{enumerate}
\item Planlegg hva som skal stå på tavla når du er ferdig.
\item Lag egene oppgaver til elevene.
\end{enumerate}




\section{Analyse av undervisningsopplegget i teori og utførelse}
\label{sec-3}



\section{Refleksjon}
\label{sec-4}
% Emacs 25.2.2 (Org mode 8.2.10)
\newpage
\bibliography{semesteroppgave}
\end{document}